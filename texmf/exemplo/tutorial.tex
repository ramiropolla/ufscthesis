%%%%%%%%%%%%%%%%%%%%%%%%%%%%%%%%%%%%%%%%%%%%%%%%%%%%%%%%%%%%%%%%%%%%%%%
% Universidade Federal de Santa Catarina             
% Biblioteca Universitária                     
%                                                           
% (c)2010 Roberto Simoni (roberto.emc@gmail.com)
%         Carlos R Rocha (cticarlo@gmail.com)
%%%%%%%%%%%%%%%%%%%%%%%%%%%%%%%%%%%%%%%%%%%%%%%%%%%%%%%%%%%%%%%%%%%%%%%
%\PassOptionsToPackage{abnt-etal-cite=1, abnt-etal-list=0}{abntcite}
\documentclass{ufscThesis}

%%%%%%%%%%%%%%%%%%%%%%%%%%%%%%%%%%%%%%%%%%%%%%%%%%%%%%%%%%%%%%%%%%%%%%%
% Pacotes usados especificamente para este documento
% Definidos pelo criador do documento
%%%%%%%%%%%%%%%%%%%%%%%%%%%%%%%%%%%%%%%%%%%%%%%%%%%%%%%%%%%%%%%%%%%%%%%
\usepackage{graphicx}

%\renewcommand{\theequation}{\arabic{equation}} %se desejar tirar o capitulo

%\usepackage[labelsep=period]{caption} % O separador de legenda é um .
\usepackage[labelsep=endash]{caption} % O separador de legenda é um -

%%%%%%%%%%%%%%%%%%%%%%%%%%%%%%%%%%%%%%%%%%%%%%%%%%%%%%%%%%%%%%%%%%%%%%%
% Identificadores do trabalho
% Usados para preencher os elementos pré-textuais
%%%%%%%%%%%%%%%%%%%%%%%%%%%%%%%%%%%%%%%%%%%%%%%%%%%%%%%%%%%%%%%%%%%%%%%
\titulo{Elaboração de documentos para a BU/UFSC} % Titulo do trabalho
\subtitulo{Estilo \LaTeX~ padrão}                % Subtitulo do trabalho (opcional)
\autor{Roberto Simoni, Carlos R Rocha}           % Nome do autor
\data{01}{julho}{2010}                           % Data da publicação do trabalho

\orientador{Prof. Dr. Fulano}                    % Nome do orientador e (opcional) seu título
\coorientador{Prof. Dr. Beltrano}                % Nome do coorientador e seu título (opcional)
\coordenador{Prof. Chefe, Dr. Eng.}              % Nome do coordenador do curso e (opcional) seu título

%\departamento[a]{Faculdade de Ciências do Mar}
%\curso[a]{Atividade de Extensão em Corte e Costura}


%%% Sobre a Banca
\numerodemembrosnabanca{5} % Isso decide se haverá uma folha adicional
\orientadornabanca{sim} % Se faz parte da banca definir como sim
\coorientadornabanca{sim} % Se faz parte da banca definir como sim
\bancaMembroA{Prof. Presidente da banca} %Nome do presidente da banca
\bancaMembroB{Prof. segundo membro}      % Nome do membro da Banca
\bancaMembroC{Prof. terceiro membro}     % Nome do membro da Banca
\bancaMembroD{Prof. quarto membro}       % Nome do membro da Banca
\bancaMembroE{Prof. quinto membro}       % Nome do membro da Banca
\bancaMembroF{Prof. sexto membro}        % Nome do membro da Banca
\bancaMembroG{Prof. sétimo membro}       % Nome do membro da Banca

\dedicatoria{A quem o trabalho é dedicado, se é que o é (opcional)}

\agradecimento{Agradecimentos opcionais, caso existam pessoas ou entidades a quem se deve apoio ou suporte ao trabalho ora apresentado.}

\epigrafe{Um bonito pensamento ou citação, se for o caso}{autor do pensamento}

\textoResumo {Aqui é redigido o resumo do documento...  blabla blablablabla blabla ipsum loren e a sophia também blab ablablabl ablbalbalblab lablablbalb lab lab lab labl a blab lablablab la blab alballbalba lba lba }

\palavrasChave {chave 1. chave 2. ... chave n.}

\textAbstract {Here is written the abstract of the document}

\keywords {key 1. key 2. ... key n.}

%%%%%%%%%%%%%%%%%%%%%%%%%%%%%%%%%%%%%%%%%%%%%%%%%%%%%%%%%%%%%%%%%%%%%%%
% Início do documento                                
%%%%%%%%%%%%%%%%%%%%%%%%%%%%%%%%%%%%%%%%%%%%%%%%%%%%%%%%%%%%%%%%%%%%%%%
\begin{document}
%--------------------------------------------------------
% Elementos pré-textuais
\capa  
\folhaderosto[comficha] % Se nao quiser imprimir a ficha, é só não usar o parâmetro
\folhaaprovacao
\paginadedicatoria
\paginaagradecimento
\paginaepigrafe
\paginaresumo
\paginaabstract
\listadefiguras
\listadetabelas 
\listadeabreviaturas
\listadesimbolos
\sumario

%-------------------------------------------------------------------------------
% Para listagens de algoritmos e de código, recomenda-se consultar os
% pacotes algorithms e lstlistings, que são usados para definir esses
% dois tipos de elementos de texto e possuem os comandos
% \listofalgorithms e \lstlistoflistings, respectivamente.
%-------------------------------------------------------------------------------

%--------------------------------------------------------
% Elementos textuais

\chapter{Introdução}
Os Anais do CONEM 2010 serão publicados em CDROM, usando o formato Adobe$^{TM}$ PDF.

Os artigos devem ser rigorosamente formatados de acordo com estas instruções e este arquivo texto pode ser usado como um template por usuários do Microsoft Word$^{TM}$ e, em qualquer caso, como um modelo para os usuários de outros softwares processadores de texto.

Os artigos estão limitados a um máximo de 10 páginas, incluindo tabelas e figuras. O arquivo final em formato pdf não deve exceder 2,5 MB.

A língua oficial do congresso é o Português, entretanto serão aceitos manuscritos em Espanhol ou em Inglês. Se o trabalho não for escrito em inglês, o autor deverá incluir o título, os nomes dos autores e afiliações, o resumo e as palavras-chave, traduzidos para o inglês, após a lista de referências, no fim do artigo.

\section{Teste2}
Texto de seção para teste

\subsection{Teste3}
Texto de subseção para teste
\begin{citacao}
 Este é um exemplo de citação. Só utilize este ambiente se
 a sua citação tiver mais de 3 linhas.
\end{citacao}


\subsubsection{Teste4}
Texto de subsubseção para teste

\paragraph{Teste5}
Texto de subsubsubseção para teste

\paragraph{Teste6}

aqui segue o barco do parágrafo normal


\chapter{Formato do Texto}
O artigo deve ser digitado em papel tamanho A4, usando a Fonte Times New Roman, tamanho 10, exceto para o título, nome de autores, instituição, endereço, resumo e palavras-chave, que têm formatações específicas indicadas acima. Espaço simples entre linhas deve ser usado ao longo do texto.

O corpo de texto que contém o título deve ser centralizado, em parágrafo com recuo esquerdo de 0,1 cm e marcado com borda esquerda de largura 2 $\frac{1}{4}$ pontos.

O corpo de texto que contém os nomes de autores e de instituições devem ser alinhados à esquerda, em parágrafo com recuo esquerdo de 0,1 cm e marcados com borda esquerda de largura 2 $\frac{1}{4}$ pontos.

A primeira página tem margem superior igual a 5 cm, e todas as outras margens (esquerda, direita e inferior) iguais a 2 cm. Todas as demais páginas do trabalho devem ter todas as suas margens iguais a 2 cm.

\section{Títulos e Subtítulos das Seções - Isso pode feder de uma forma nunca vista, se tu não prestares atenção}
Os títulos e subtítulos das seções devem ser digitados em formato Times New Roman, tamanho 10, estilo negrito, e alinhados à esquerda. Os títulos das seções são com letras maiúsculas (Exemplo: \textbf{MODELO MATEMÁTICO}), enquanto que os subtítulos só têm as primeiras letras maiúsculas (Exemplo: \textbf{Modelo Matemático}). Eles devem ser numerados, usando numerais arábicos separados por pontos, até o máximo de 3 subníveis. Uma linha em branco de espaçamento simples deve ser incluída acima e abaixo de cada título/subtítulo.

\section{Corpo do Texto}
O corpo do texto é justificado e com espaçamento simples. A primeira linha de cada parágrafo tem recuo de 0,6 cm contado a partir da margem esquerda.

As equações matemáticas são alinhadas à esquerda com recuo de 0,6 cm.  Elas são referidas por Eq. (1) no meio da frase, ou por Equação (1) quando usada no início de uma sentença. Os números das equações são numerais arábicos colocados entre parênteses, e alinhados à direita, como mostrado na Eq. (1).

Os símbolos usados nas equações devem ser definidos imediatamente antes ou depois de sua primeira ocorrência no texto do trabalho.

O tamanho da fonte usado nas equações deve ser compatível com o utilizado no texto. Todos os símbolos devem ter suas unidades expressas no sistema S.I. (métrico).
\abreviatura{SI}{Sistema Internacional de unidades}
\begin{equation}
\frac{\partial ^2 T}{\partial x^2} + \frac{\partial ^2 T}{\partial y^2} =0
\end{equation}
\simbolo{x,y}{Coordenadas do plano cartesiano}
\simbolo{T}{Tensão}

As tabelas devem ser centralizadas. Elas são referidas por Tab. (1) no meio da frase, ou por Tabela (1) quando usada no início de uma sentença. Sua legenda é centralizada e localizada imediatamente acima da tabela. Anotações e valores numéricos nela incluídos devem ter tamanhos compatíveis com o da fonte usado no texto do trabalho, e todas as unidades devem ser expressas no sistema S.I. (métrico). As unidades são incluídas apenas na primeira linha ou primeira coluna de cada tabela, conforme for apropriado. As tabelas devem ser colocadas tão perto quanto possível de sua primeira citação no texto. Deixe uma linha simples em branco entre a tabela, seu título e o texto.

O estilo de borda da tabela é livre. As legendas das Figuras e das Tabelas não devem exceder 3 linhas.

\begin{table}[h]
\begin{center}
\caption{Exemplo de tabela}
\begin{tabular}{c|c|c}
\hline
Propriedades do compósito & CFRC-TWILL & CFRC-4HS\\
\hline
Resistência à Flexão (MPa) & 209$\pm$ 10 & 180 $\pm$  15\\
\hline
Módulo de Flexão  (GPa) & 57.0 $\pm$ 2.8 & 18.0 $\pm$  1.3\\
\hline
\end{tabular}
\end{center}
\end{table}

As figuras são centralizadas. Elas são referenciadas por Fig. (1) no meio da frase ou por Figura (1) quando usada no início de uma sentença. Sua legenda é centralizada e localizada imediatamente abaixo da figura. As anotações e numerações devem tem tamanhos compatíveis com o da fonte usada no texto, e todas as unidades devem ser expressas no sistema S.I. (métrico). As figuras devem ser colocadas o mais próximo possível de sua primeira citação no texto. Deixe uma linha em branco entre as figuras e o texto.

\begin{figure}[ht]
\begin{center}
\includegraphics[scale=0.6]{figuras/figura.jpg}
\caption{Exemplo de figura com legenda bem grande para ver o problema que pode ocorrer nas indentações. Espero que seja o suficiente.}
\end{center}
\end{figure}

Figuras coloridas e fotografias de alta qualidade podem ser incluídas no trabalho. Para reduzir o tamanho do arquivo e preservar a resolução gráfica, converta os arquivos das imagens para o  formato GIFF (para figuras com até 16 cores) ou para o formato JPEG (alta densidade de cores), antes de inseri-los no trabalho.
\sigla{GNU}{Gnu is Not Unix}
A citação das referências no corpo do texto pode ser feita nos formatos: \citeonline{Bordalo02}, mostra que o corpo..., ou: Vários trabalhos  \cite{Coimbra84,Clark86,Sparrow80} mostram que a rigidez da viga.

Referências aceitas incluem: artigos de periódicos \cite{Soviero97}, dissertações, teses \cite{Lee03}, artigos publicados em anais de congressos, livros, comunicações privadas, publicações na web \cite{ABCM04,MLA04} e artigos submetidos e aceitos (identificar a fonte)\cite{Autor04}.

A lista de referências é uma nova seção denominada Referências, localizada no fim do artigo.

A primeira linha de cada referência é alinhada à esquerda; todas as outras linhas têm recuo de 0,6 cm da margem esquerda. Todas as referências incluídas na lista devem aparecer como citações no texto do trabalho.

As referências devem ser postas < > em ordem alfabética, usando o último nome do primeiro autor, seguida do ano da publicação. Exemplo da lista de referências é apresentado abaixo \cite{Rocha-2010-Teste}.

\chapter{Agradecimentos}
Esta seção, se houver, deve ser colocada antes da lista de referências.

\chapter{Direitos Autorais}
Os autores são os únicos responsáveis pelo conteúdo do material impresso incluído no seu trabalho.

\bibliographystyle{ufscThesis/ufsc-alf}
\bibliography{bibliografia}

%--------------------------------------------------------
% Elementos pós-textuais

\apendice
\chapter{Teste Apêndice}
\section{Teste apêndice seção - como as coisas podem desandar quando não olhamos direito para os detalhes}
teste

blablabla

\anexo
\chapter{Teste Anexo}
conteúdo do anexo

\end{document}
%%%%%%%%%%%%%%%%%%%%%%%%%%%%%%%%%%%%%%%%%%%%%%%%%%%%%%%%%%%%%%%%%%%%%%%
% Fim do documento                                
%%%%%%%%%%%%%%%%%%%%%%%%%%%%%%%%%%%%%%%%%%%%%%%%%%%%%%%%%%%%%%%%%%%%%%%



%%%%%%%%%%%%%%%%%%%%%%%%%%%%%%%%%%%%%%%%%%%%%%%%%%%%%%%%%%%%%%%%%%%%%%%
% Identificadores do trabalho
% Usados para preencher os elementos pré-textuais
%%%%%%%%%%%%%%%%%%%%%%%%%%%%%%%%%%%%%%%%%%%%%%%%%%%%%%%%%%%%%%%%%%%%%%%

%----------------------------------------------------------------------
% Só preencher se não for a UFSC - Se for uma instituição "masculina",
% como um Instituto Federal, usar o parâmetro opcional [] - v. exemplo
%
%\instituicao[o]{Instituto Federal do Rio Grande do Sul}

%----------------------------------------------------------------------
% Só preencher se não for o departamento de Eng. Mecânica - o que deve 
% ser quase que certo. Se for um departamento "feminino", usar o
% parâmetro opcional [] - v. exemplo
%
%\departamento[a]{Faculdade de Ciências do Mar}

%----------------------------------------------------------------------
% Só preencher se não for o POSMEC - o que deve ser quase que certo.
% Se for um curso "feminino", usar o parâmetro opcional [] - v. exemplo
%
%\curso[a]{Atividade de Extensão em Corte e Costura}

%----------------------------------------------------------------------
% Só preencher se não for tese
% Se for um documento diferente de tese, dissertação, tcc, monografia
% ou relatório, indicar no parâmetro opcional o gênero - v. exemplo
%
%\documento[o]{Laudo}

%----------------------------------------------------------------------
% Título é obrigatório, mas subtítulo é opcional
%
%\titulo{Elaboração de documentos para a BU/UFSC}
%\subtitulo{Estilo \LaTeX~ Padrão}

%----------------------------------------------------------------------
% Autor é obrigatório. Não se atreva a não incluir ou vai ter surpresa
%
%\autor{Roberto Simoni, Carlos R Rocha}

%----------------------------------------------------------------------
%
% Só preencher se não for Doutor em Engenharia Mecânica
%\grau{Descomentar se não for Doutor em Engenharia Mecânica}

%----------------------------------------------------------------------
% Só preencher se não for Florianópolis
%
%\local{Simcity}

%----------------------------------------------------------------------
% Data deve ter as três partes entre chaves
%
%\data{01}{julho}{2010}

%----------------------------------------------------------------------
% Orientador é obrigatório. Coorientador é opcional
% Se o título for diferente (orientadora), indicar como no exemplo
%
%\orientador[Orientadora]{Profa. Dra. Fulana}
%\coorientador{Prof. Dr. Beltrano}

%----------------------------------------------------------------------
% Coordenador do programa é obrigatório
% Se o título for diferente (coordenadora), indicar como no exemplo
%
%\coordenador[Coordenadora]{Profa. Senhora, Dra. Eng.}

%----------------------------------------------------------------------
% Banca - Pode ter até 7 membros além de orientador e co-orientador
% Se estes são parte da banca, devem ser adicionados com os comandos
% \orientadornabanca{sim} e \coorientadornabanca{sim}
% do contrário, eles aparecerão antes da designação da banca
% O MembroA da banca é por definição o seu presidente
% O numero total de membros na defesa decide se a folha de aprovação
% deverá ser duplicada. Se passar de 4, uma folha adicional de assinaturas
% será gerada
%
%\numerodemembrosnabanca{4} % Isso decide se haverá uma folha adicional
%\orientadornabanca{sim} % Se faz parte da banca definir como sim
%\coorientadornabanca{sim} % Se faz parte da banca definir como sim
%\bancaMembroA{Prof. Presidente da banca} %Nome do presidente da banca
%\bancaMembroB{Prof. segundo membro}      % Nome do membro da Banca
%\bancaMembroC{Prof. terceiro membro}     % Nome do membro da Banca
%\bancaMembroD{Prof. quarto membro}       % Nome do membro da Banca
%\bancaMembroE{Prof. quinto membro}       % Nome do membro da Banca
%\bancaMembroF{Prof. sexto membro}        % Nome do membro da Banca
%\bancaMembroG{Prof. sétimo membro}       % Nome do membro da Banca

%----------------------------------------------------------------------
% Firulas opcionais - Dedicatória, Agradecimento e Epígrafe
%
% \dedicatoria{Dedicatória para alguem}
% \agradecimento{Agradecimentos, se for o caso...blabla blablablabla blabla ipsum loren e a sophia também blab ablablabl ablbalbalblab lablablbalb lab la}
% \epigrafe{Um bonito pensamento ou citação, se for o caso}{autor do pensamento}

%----------------------------------------------------------------------
% Resumo e abstract - É só definir como mostra o exemplo abaixo
% 
% \textoResumo {Aqui é redigido o resumo do documento...  blabla blablablabla blabla ipsum loren e a sophia também blab ablablabl ablbalbalblab lablablbalb lab lab lab labl a blab lablablab la blab alballbalba lba lba }
% \palavrasChave {chave 1. chave 2. ... chave n.}
% 
% \textAbstract {Here is written the abstract of the document}
% \keywords {key 1. key 2. ... key n.}
